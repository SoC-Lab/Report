\chapter{Conclusion}
This work has explored the possibility of using partial reconfiguration in \glspl{FPGA} to provide efficient redundancy in an automotive system.
Instead of providing a redundant hardware entity for every critical module, one single \gls{FPGA} provides dynamic redundancy for each of these modules.
To avoid over-commitment with regards to resource usage (space and power) partial reconfiguration is used.
By using the newly available Cortex M1 IPs, a streamlined software development process is possible. 
The same software can be executed on the cores in the \gls{FPGA} as well as on the actual hardware with minimal adaptions in the build-process.
This reduces the amount of testing and tool-chain adaptions that need to be performed.
We demonstrated these concepts on the Zynq-7000 and with three Cortex M1 \glspl{CPU} that were connected with a bus.

\section{Future Work}
Based on this work a more heterogeneous set of critical hardware could be provided with redundancy. 
A good first step would be to include the Cortex M3 CPU that couldn't be included in this project due to time constraints.
The bus monitor could be extended with a more sophisticated fault detection algorithm, which could also mean to employ a more sophisticated bus protocol.
Measurements with regards to the system performance (e.g. time to reconfigure, time to detect fault, time to mitigate fault, power usage ...) should be considered also.
Furthermore the Mbed OS port could be extended by other hardware modules. A first draft for the Timer and I2C modules ia already implemented.